%-------------------------------------------------------------------------------
%	SECTION TITLE
%-------------------------------------------------------------------------------
\cvsection{Work Experience}


%-------------------------------------------------------------------------------
%	CONTENT
%-------------------------------------------------------------------------------
\begin{cventries}
    \cventry
        {Technical Leader} % Title
        {Mar. 2020 - Present} % Date(s)
        {Finance Service of NAVER FINANCIAL Corp. in NAVER Corp., } % Organization
        {Sungnam, Korea}% Location
        { % Corp desc
            NAVER FINANCIAL operates financial platform including online payments, loan services from Nov. 2019, and the number of monthly active users is over 13 million in Mar. 2021.
        }
        { % Description(s) of tasks/responsibilities
            \begin{cvitems}
                \item {
                    {\bf Data Pipeline} \\
                    - Building and sustaining a secure Hadoop environment \\
                        \descstyle{The Hadoop has about two hundreds user}
                        \techstyle{Hadoop, Kerberos, Ranger, Ansible, OpenWhisk} \\
                    - Automatically collecting the massive and various formatted data from diverse data sources \\
                        \descstyle{Only configurations are required when ingesting new data without source codes}
                        \techstyle{Hive 3.0, Compactor, CDC (Change Data Capture), OracleDB, MySQL, MongoDB, Hive, Kafka Connect, Airflow, Spark} \nn
                }
                \item {
                    {\bf Data Science} \\
                    - For target marketing, identifying users on specific conditions such as vehicle owners, potential customers, and so on \\
                        \descstyle{We profile users for preparing diverse finance services}
                        \techstyle{LightGBM, Lift Score, Bayesian Optimization} \\
                    - Analyzing financial and commercial data for promotion activities of NAVER FINANCIAL services \\
                        \techstyle{Propensity Score Analysis, Casual Discovery} \nn
                }
            \end{cvitems}
        }

    \cventry
        {Technical Leader} % Title
        {Dec. 2018 - Mar. 2020} % Date(s)
        {FLO Music Service of Dreamus Company in SK Telecom, } % Organization
        {Seoul, Korea}% Location
        { % Corp desc
            SK Telecom is the largest mobile operator in Korea, and FLO music service has 1.3 million memberships.
        }
        { % Description(s) of tasks/responsibilities
            \begin{cvitems}
                \item {
                    {\bf Infrastructure Developments} \\
                    - Migration to AWS from on-premise infrastructure \\
                        \descstyle{We want to get scalable big data platform, and now, under 100 engineers are using this infrastructure}
                        \techstyle{EC2, VPC, S3, EMR, CloudWatch} \\
                    - Gathering and visualizing data from multiple sources \\
                        \descstyle{Now, we measure the right KPIs on our dashboards}
                        \techstyle{EMR, Aurora, Presto, Redash, openLDAP} \nn
                }
                \item {
                    {\bf Music Recommender System} \\
                    - Design/development a data pipeline including log data and music metadata \\
                        \descstyle{There are about 10 GB of daily log data and 20 GB of music metadata}
                        \techstyle{S3, Glue, Hive, Spark, Presto, Sqoop, Docker} \\
                    - Personalized playlist generation with trained models \\
                        \descstyle{27 percentages of users listen with these playlists in Aug. 2019, though under 5 percentages of users listened in Jan. 2019}
                        \techstyle{Fasttext, Faiss, Spark} \\
                    - Continuous extraction of mood features from songs with music information retrieval techniques \\
                        \descstyle{Mood information has been used for the recommendation based on users' mood preference}
                        \techstyle{Tensorflow, Lambda, ECS, SQS, DynamoDB, S3, Glue} \nn
                }
            \end{cvitems}
        }

    \cventry
        {Software Engineer} % Title
        {Aug. 2015 - Nov. 2018} % Date(s)
        {Music Streaming Service in NAVER Corp., } % Organization
        {Sungnam, Korea}% Location
        { % Corp desc
            NAVER is Korea's top search portal, and NAVER music streaming service has a membership of over 0.8 million in 2018.
        }
        { % Description(s) of tasks/responsibilities
            \begin{cvitems}
                \item {
                    {\bf Data Pipeline} \\
                    - Mining and cleansing the massive data in Hadoop environments \\
                        \descstyle{About 1 GB of daily log data and 10 GB of music metadata}
                        \techstyle{Python, Spark, Hive, Presto} \\
                    - Development of a pipeline to extract mood features from song \\
                        \descstyle{That pipeline extract those features as soon as a song is released newly}
                        \techstyle{Python, PyArrow, RabbitMQ, Nomad, Consul} \\
                    - Design/development of A/B testing platform 
                        \techstyle{HBase, Spark, SpringBoot} \nn
                }
                \item {
                    {\bf Music Recommender System} \\
                    - Item recommendation of similarity search among millions of items while maintaining up-to-date information \\
                        \descstyle{These songs and artists have been used to recommend preferable songs with collaborative filtering}
                        \techstyle{Gensim, Spark MLLib, Fasttext, Annoy, Faiss} \\
                    - Modeling musical instrument classifier with audio content data and convolutional neural networks \\
                        \descstyle{Instrumental songs such as orgel songs have been recommend in special cases}
                        \techstyle{Keras, MIR (Music Information Retrieval), CNN} \\
                    - Development of a real-time recommender system (ratio station) applying user's listening feedback \\
                        \descstyle{About 20 percentage of users were listening with this continuous ratio station in Dec. 2018}
                        \techstyle{Python, Redis, Similarity Search, Matrix Factorization} \\
                    - Recommendation songs not listened yet applying a mixture of collaborative/content-based filtering \\
                        \descstyle{About 10 percentage of users were listening discovered songs in Dec. 2018}
                        \techstyle{Spark, Hive, Matrix Factorization} \nn
                }
            \end{cvitems}
        }
\end{cventries}