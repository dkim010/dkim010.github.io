%-------------------------------------------------------------------------------
%	SECTION TITLE
%-------------------------------------------------------------------------------
\cvsection{Work Experience}


%-------------------------------------------------------------------------------
%	CONTENT
%-------------------------------------------------------------------------------
\begin{cventries}
    \cventry
        {Senior Software Engineer} % Title
        {May 2022 - Present} % Date(s)
        {Toss of Viva Republica, } % Organization
        {Seoul, Korea}% Location
        { % Corp desc
            Viva Republica operates financial super app Toss including payment services, financial services, and stock brokerage with over 15 million active users in Sep. 2023.
        }
        { % Description(s) of tasks/responsibilities
            \begin{cvitems}
                \item {
                    {\bf Machine Learning Platform} \\
                    - Improving the development culture of my team \\
                        \descstyle{Bringing together projects where individuals were working on different roadmaps, different code styles, and different patterns so that they were unified and on the same page.} \\
                    - A DAG factory framework to easily develop DAGs for big data processing without data engineering knowledge \\
                        \descstyle{Many Airflow operators are supported like Redis operator with YAML-formatted configurations for low code.}
                        \techstyle{Airflow, Impala, Spark, Kubernetes} \\
                    - An in-house feature store which utilizes a variety of storage to store and retrieve features for offline/online environments \\
                        \descstyle{It is being used by ad services for real-time inference.}
                        \techstyle{Spark, Kubernetes, HBase, Aerospike, Kotlin, Spring} \\
                    - A recommendation platform that allows service providers to define the scoring formula based on various types of items \\
                        \techstyle{Spark, Redis, Kotlin, Spring} \\
                    - Design/Developed Multi-armed Bandits platforms for recommendations of shopping. \\
                        \techstyle{Spark, Redis, HBase} \nn
                }
            \end{cvitems}
        }

    \cventry
        {Technical Leader} % Title
        {Mar. 2020 - May 2022} % Date(s)
        {Finance Service of NAVER FINANCIAL Corp. in NAVER Corp., } % Organization
        {Sungnam, Korea}% Location
        { % Corp desc
            NAVER FINANCIAL operates financial platform including online payments, loan services from Nov. 2019, and the number of monthly active users is over 13 million in Mar. 2021.
        }
        { % Description(s) of tasks/responsibilities
            \begin{cvitems}
                \item {
                    {\bf Data Pipeline / Analysis} \\
                    - Building and sustaining a secure Hadoop environment \\
                        \descstyle{The Hadoop has about two hundreds user}
                        \techstyle{Hadoop, Kerberos, Ranger, Ansible, OpenWhisk} \\
                    - Automatically collecting the massive and various formatted data from diverse data sources \\
                        \descstyle{Only configurations are required when ingesting new data without source codes}
                        \techstyle{Hive, Compactor, OracleDB, MySQL, MongoDB, Hive, Kafka Connect, Airflow, Spark} \\
                    - For target marketing, identifying users on specific conditions such as vehicle owners, potential customers, and so on \\
                        \descstyle{We profile users for preparing diverse finance services}
                        \techstyle{LightGBM, Lift Score, Bayesian Optimization} \\
                    - Analyzing financial and commercial data for promotion activities of NAVER FINANCIAL services \\
                        \techstyle{Propensity Score Analysis, Casual Discovery} \nn
                }
            \end{cvitems}
        }

    \cventry
        {Technical Leader} % Title
        {Dec. 2018 - Mar. 2020} % Date(s)
        {FLO Music Service of Dreamus Company in SK Telecom, } % Organization
        {Seoul, Korea}% Location
        { % Corp desc
            SK Telecom is the largest mobile operator in Korea, and FLO music service has 1.3 million memberships.
        }
        { % Description(s) of tasks/responsibilities
            \begin{cvitems}
                \item {
                    {\bf Music Recommender System} \\
                    - Migration to AWS from on-premise infrastructure \\
                        \descstyle{We want to get scalable big data platform, under 100 engineers were using this infrastructure}
                        \techstyle{EC2, VPC, S3, EMR, Aurora, Presto, Redash, CloudWatch} \\
                    - Design/development a data pipeline including log data and music metadata \\
                        \descstyle{There are about 10 GB of daily log data and 20 GB of music metadata}
                        \techstyle{S3, Glue, Hive, Spark, Presto, Sqoop, Docker} \\
                    - Personalized playlist generation with trained models \\
                        \descstyle{27 percentages of users listen with these playlists in Aug. 2019, though under 5 percentages of users listened in Jan. 2019}
                        \techstyle{Fasttext, Faiss, Spark} \\
                    - Continuous extraction of mood features from songs with music information retrieval techniques \\
                        \descstyle{Mood information has been used for the recommendation based on users' mood preference}
                        \techstyle{Tensorflow, Lambda, ECS, SQS, DynamoDB, S3, Glue} \nn
                }
            \end{cvitems}
        }

    \cventry
        {Software Engineer} % Title
        {Aug. 2015 - Nov. 2018} % Date(s)
        {Music Streaming Service in NAVER Corp., } % Organization
        {Sungnam, Korea}% Location
        { % Corp desc
            NAVER is Korea's top search portal, and NAVER music streaming service has a membership of over 0.8 million in 2018.
        }
        { % Description(s) of tasks/responsibilities
            \begin{cvitems}
                \item {
                    {\bf Music Recommender System} \\
                    - Development of a pipeline to extract mood features from song \\
                        \descstyle{That pipeline extract those features as soon as a song is released newly}
                        \techstyle{Python, PyArrow, RabbitMQ, Nomad, Consul} \\
                    - Development of a real-time recommender system (ratio station) applying user's listening feedback \\
                        \descstyle{About 20 percentage of users were listening with this continuous ratio station in Dec. 2018}
                        \techstyle{Python, Redis, Similarity Search, Matrix Factorization, Gensim, Fasttext, Annoy, Faiss} \\
                    - Recommendation songs not listened yet applying a mixture of collaborative/content-based filtering \\
                        \descstyle{About 10 percentage of users were listening discovered songs in Dec. 2018}
                        \techstyle{Spark, Hive, Matrix Factorization} \\
                    - Design/development of A/B testing platform 
                        \techstyle{HBase, Spark, SpringBoot} \nn
                }
            \end{cvitems}
        }
\end{cventries}
