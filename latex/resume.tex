%!TEX TS-program = xelatex
%!TEX encoding = UTF-8 Unicode
% Awesome CV LaTeX Template for CV/Resume
%
% This template has been downloaded from:
% https://github.com/posquit0/Awesome-CV
%
% Author:
% Claud D. Park <posquit0.bj@gmail.com>
% http://www.posquit0.com
%
% Template license:
% CC BY-SA 4.0 (https://creativecommons.org/licenses/by-sa/4.0/)
%


%-------------------------------------------------------------------------------
% CONFIGURATIONS
%-------------------------------------------------------------------------------
% A4 paper size by default, use 'letterpaper' for US letter
\documentclass[12pt, a4paper]{awesome-cv}

% Configure page margins with geometry
% \geometry{left=1.4cm, top=1.0cm, right=1.4cm, bottom=1.8cm, footskip=.5cm}
% \geometry{left=1.6cm, top=1.0cm, right=1.6cm, bottom=1.8cm, footskip=.5cm}
\geometry{left=1.6cm, top=1.7cm, right=1.6cm, bottom=1.8cm, footskip=.5cm}

% Specify the location of the included fonts
\fontdir[fonts/]

% Color for highlights
% Awesome Colors: awesome-emerald, awesome-skyblue, awesome-red, awesome-pink, awesome-orange
%                 awesome-nephritis, awesome-concrete, awesome-darknight
\colorlet{awesome}{awesome-red}
% Uncomment if you would like to specify your own color
% \definecolor{awesome}{HTML}{CA63A8}

% Colors for text
% Uncomment if you would like to specify your own color
% \definecolor{darktext}{HTML}{414141}
% \definecolor{text}{HTML}{333333}
% \definecolor{graytext}{HTML}{5D5D5D}
% \definecolor{lighttext}{HTML}{999999}

% Set false if you don't want to highlight section with awesome color
\setbool{acvSectionColorHighlight}{true}

% If you would like to change the social information separator from a pipe (|) to something else
\renewcommand{\acvHeaderSocialSep}{\quad\textbar\quad}

\makeatletter
\patchcmd{\@sectioncolor}{\color}{\mdseries\color}{}{}
\makeatother

%-------------------------------------------------------------------------------
%	PERSONAL INFORMATION
%	Comment any of the lines below if they are not required
%-------------------------------------------------------------------------------
% Available options: circle|rectangle,edge/noedge,left/right
% \photo[rectangle,edge,right]{profile}
\name{}{DONGWON KIM}
% \position{Software Engineer{\enskip\cdotp\enskip}Security Expert}
\position{Software Engineer}
% \address{Sungnam, Korea}

\mobile{} %FIXME
\email{dkim010@gmail.com}
% \homepage{dkim010.github.io}
\github{dkim010}
\linkedin{dongwon-kim-68235a35}
\googlescholar{37h1QgcAAAAJ}
% \gitlab{gitlab-id}
% \stackoverflow{SO-id}{SO-name}
% \twitter{@twit}
% \skype{skype-id}
% \reddit{reddit-id}
% \extrainfo{extra informations}

% \quote{``Be the change that you want to see in the world."}


%-------------------------------------------------------------------------------
\begin{document}

% Print the header with above personal informations
% Give optional argument to change alignment(C: center, L: left, R: right)
\makecvheader[C]

% Print the footer with 3 arguments(<left>, <center>, <right>)
% Leave any of these blank if they are not needed
% \makecvfooter
%   {\today}
%   {Claud D. Park~~~·~~~Résumé}
%   {\thepage}


%-------------------------------------------------------------------------------
%	CV/RESUME CONTENT
%	Each section is imported separately, open each file in turn to modify content
%-------------------------------------------------------------------------------


% %-------------------------------------------------------------------------------
%	SECTION TITLE
%-------------------------------------------------------------------------------
\cvsection{Summary}


%-------------------------------------------------------------------------------
%	CONTENT
%-------------------------------------------------------------------------------
\begin{cventries}
    \cvsimpleentry
        { % Description(s) bullet points
            \begin{cvitems}
                \item{
                    {\bf Career} \\
                    - Data engineer for a music streaming service \\
                    \techstyle{Understanding the whole processes of music recommendation} \\
                    - System engineer for infrastructures \\
                    \techstyle{Understanding the internal operations of computer systems} \nn
                }
                \item{
                    {\bf Fields}: Cloud computing, data engineering, recommender system, system engineering \nn
                }
                \item{
                    {\bf Programming Languages}: Python, C, Bash, Scala, Java \nn
                }
                \item{
                    {\bf Skills}: Spark, MapReduce, Hive, Presto, HBase, Consul, Nomad, Docker, RabbitMQ, FastText, Annoy, Faiss\nn
                }
                % \item{
                %     {\bf Hobby}: League of Legends, playing acoustic guitar, listening to music
                % }
            \end{cvitems}
        }
\end{cventries}
% %-------------------------------------------------------------------------------
%	SECTION TITLE
%-------------------------------------------------------------------------------
\cvsection{Work Experience}


%-------------------------------------------------------------------------------
%	CONTENT
%-------------------------------------------------------------------------------
\begin{cventries}
    \cventry
        {Music Recommender System} % Title
        {Oct. 2016 - Present} % Date(s)
        {NAVER Corp.,} % Organization
        {Sungnam, Korea}% Location
        { % Corp desc
            Korea's top search portal which of the music service has a membership of over 1.2 million.
        }
        { % Description(s) of tasks/responsibilities
            \begin{cvitems}
                \item {
                    {\bf Data Pipeline} \\
                    - Collecting and cleansing the massive data in Hadoop environments
                        \techstyle{Python, Spark, Hive} \\
                    - Development of a pipeline for audio feature extraction with various trained models
                        \techstyle{Python, Pyarrow, RabbitMQ} \nn
                }
                \item {
                    {\bf Item-to-item Recommendation} \\
                    - Song and artist recommendation which you may also like
                        \techstyle{Gensim, Spark MLLib, Fasttext} \\
                    - Similarity search between millions of items while maintaining up-to-date information
                        \techstyle{Annoy, Faiss, Spark} \\
                    - Modeling musical instrument classifier with audio content data and convolutional neural networks
                        \techstyle{Keras} \\
                    - Song recommendation from natural language based on word embeddings
                        \techstyle{Gensim, Fasttext} \\
                    - Discarding duplicate songs among millions of songs using titles and audio contents
                        \techstyle{Spark, Hive} \nn
                }
                \item {
                    {\bf Playlist Generation \& Continuation} \\
                    - Discover songs not listened yet applying a mixture of collaborative/content-based filtering
                        \techstyle{Spark, Hive} \\
                    - Development of a real-time recommender system applying user's listening feedback
                        \techstyle{Python, Redis} \nn
                }
                \item{
                    {\bf Infrastructure Developments} \\
                    - Design and development of A/B test platform
                        \techstyle{HBase, Spark, Spring Boot} \nn
                }
            \end{cvitems}
        }
\end{cventries}
% %-------------------------------------------------------------------------------
%	SECTION TITLE
%-------------------------------------------------------------------------------
\cvsection{Education}


%-------------------------------------------------------------------------------
%	CONTENT
%-------------------------------------------------------------------------------
\begin{cventries}
    \cveducationentry
        {M.Sc. in Computer Science} % Degree
        {Mar. 2011 - Aug. 2015} % Date(s)
        {Yonsei University, } % Institution
        {Seoul, Korea} % Location
        {one of Korea's top-three universities, considered the most prestigious in the country.} % Institution desc
        {``Content-centric energy management for mobile devices'' {\em (Advisor: Prof. Hojung Cha)}}
        {} %FIXME
        
    \cveducationentry
        {B.S. in Computer Science} % Degree
        {Mar. 2007 - Feb. 2011} % Date(s)
        {Yonsei University} % Institution
        {Seoul, Korea} % Location
        {} % Institution desc
        {} % title paper
        {} %FIXME
\end{cventries}

% %-------------------------------------------------------------------------------
%	SECTION TITLE
%-------------------------------------------------------------------------------
\cvsection{Selected Publications}


%-------------------------------------------------------------------------------
%	CONTENT
%-------------------------------------------------------------------------------
\begin{cventries}
    \cvacademicentry
        {Deep Learning (3 papers)} % Header
        {} % Proceeding
        { % Description(s) bullet points
            \begin{cvitems}
                \item {\small B. Jeon, A. Kim, C. Kim, \textbf{D. Kim}, J. Park, J. Ha, ``Music Emotion Recognition via End-to-End Multimodal Neural Networks,'' {\em Proceeding of the 11th ACM Conference on Recommender Systems} (RecSys 2017), Aug. 2017}
                % \item {\small J. Ha, A. Kim, \textbf{D. Kim}, C. Kim, J. Park, ``Music Highlight Extraction via Convolutional Recurrent Attention Networks,'' {\em Machine Learning for Music Discovery Workshop in International Conference on Machine Learning} (ICML 2017), Aug, 2017.}
            \end{cvitems}
        }

    \cvacademicentry
        {System Engineering (12 papers)} % Header
        {} % Proceeding
        { % Description(s) bullet points
            \begin{cvitems}
                \item {\small C. Yoon, \textbf{D. Kim}, W. Jung, C. Kang, H. Cha, ``AppScope: Application Energy Metering Framework for Android Smartphone using Kernel Activity Monitoring,'' {\em Proceeding of 2012 USENIX Annual Technical Conference} (USENIX ATC), Jun., 2012.}
                \item {\small \textbf{D. Kim}, N. Jung, Y. Chon, H. Cha, ``Content-Centric Energy Management of Mobile Displays,'' {\em IEEE Transactions on Mobile Computing} (TMC), DOI: 10.1109/TMC.2015.2467393}
            \end{cvitems}
        }
\end{cventries}


% %-------------------------------------------------------------------------------
%	SECTION TITLE
%-------------------------------------------------------------------------------
\cvsection{Summary}


%-------------------------------------------------------------------------------
%	CONTENT
%-------------------------------------------------------------------------------
\begin{cventries}
    \cvsimpleentry
        { % Description(s) bullet points
            \begin{cvitems}
                \item{
                    {\bf Career} \\
                    - Data engineer for a music streaming service \\
                    \techstyle{Understanding the whole processes of music recommendation} \\
                    - System engineer for infrastructures \\
                    \techstyle{Understanding the internal operations of computer systems} \nn
                }
                \item{
                    {\bf Fields}: Cloud computing, data engineering, recommender system, system engineering \nn
                }
                \item{
                    {\bf Programming Languages}: Python, C, Bash, Scala, Java \nn
                }
                \item{
                    {\bf Skills}: Spark, MapReduce, Hive, Presto, HBase, Consul, Nomad, Docker, RabbitMQ, FastText, Annoy, Faiss\nn
                }
                % \item{
                %     {\bf Hobby}: League of Legends, playing acoustic guitar, listening to music
                % }
            \end{cvitems}
        }
\end{cventries}
% %-------------------------------------------------------------------------------
%	SECTION TITLE
%-------------------------------------------------------------------------------
\cvsection{Work Experience}


%-------------------------------------------------------------------------------
%	CONTENT
%-------------------------------------------------------------------------------
\begin{cventries}
    \cventry
        {Music Recommender System} % Title
        {Oct. 2016 - Present} % Date(s)
        {NAVER Corp.,} % Organization
        {Sungnam, Korea}% Location
        { % Corp desc
            Korea's top search portal which of the music service has a membership of over 1.2 million.
        }
        { % Description(s) of tasks/responsibilities
            \begin{cvitems}
                \item {
                    {\bf Data Pipeline} \\
                    - Collecting and cleansing the massive data in Hadoop environments
                        \techstyle{Python, Spark, Hive} \\
                    - Development of a pipeline for audio feature extraction with various trained models
                        \techstyle{Python, Pyarrow, RabbitMQ} \nn
                }
                \item {
                    {\bf Item-to-item Recommendation} \\
                    - Song and artist recommendation which you may also like
                        \techstyle{Gensim, Spark MLLib, Fasttext} \\
                    - Similarity search between millions of items while maintaining up-to-date information
                        \techstyle{Annoy, Faiss, Spark} \\
                    - Modeling musical instrument classifier with audio content data and convolutional neural networks
                        \techstyle{Keras} \\
                    - Song recommendation from natural language based on word embeddings
                        \techstyle{Gensim, Fasttext} \\
                    - Discarding duplicate songs among millions of songs using titles and audio contents
                        \techstyle{Spark, Hive} \nn
                }
                \item {
                    {\bf Playlist Generation \& Continuation} \\
                    - Discover songs not listened yet applying a mixture of collaborative/content-based filtering
                        \techstyle{Spark, Hive} \\
                    - Development of a real-time recommender system applying user's listening feedback
                        \techstyle{Python, Redis} \nn
                }
                \item{
                    {\bf Infrastructure Developments} \\
                    - Design and development of A/B test platform
                        \techstyle{HBase, Spark, Spring Boot} \nn
                }
            \end{cvitems}
        }
\end{cventries}
% \pagebreak
% %-------------------------------------------------------------------------------
%	SECTION TITLE
%-------------------------------------------------------------------------------
\cvsection{Education}


%-------------------------------------------------------------------------------
%	CONTENT
%-------------------------------------------------------------------------------
\begin{cventries}
    \cveducationentry
        {M.Sc. in Computer Science} % Degree
        {Mar. 2011 - Aug. 2015} % Date(s)
        {Yonsei University, } % Institution
        {Seoul, Korea} % Location
        {one of Korea's top-three universities, considered the most prestigious in the country.} % Institution desc
        {``Content-centric energy management for mobile devices'' {\em (Advisor: Prof. Hojung Cha)}}
        {} %FIXME
        
    \cveducationentry
        {B.S. in Computer Science} % Degree
        {Mar. 2007 - Feb. 2011} % Date(s)
        {Yonsei University} % Institution
        {Seoul, Korea} % Location
        {} % Institution desc
        {} % title paper
        {} %FIXME
\end{cventries}

% %-------------------------------------------------------------------------------
%	SECTION TITLE
%-------------------------------------------------------------------------------
\cvsection{Selected Publications}


%-------------------------------------------------------------------------------
%	CONTENT
%-------------------------------------------------------------------------------
\begin{cventries}
    \cvacademicentry
        {Deep Learning (3 papers)} % Header
        {} % Proceeding
        { % Description(s) bullet points
            \begin{cvitems}
                \item {\small B. Jeon, A. Kim, C. Kim, \textbf{D. Kim}, J. Park, J. Ha, ``Music Emotion Recognition via End-to-End Multimodal Neural Networks,'' {\em Proceeding of the 11th ACM Conference on Recommender Systems} (RecSys 2017), Aug. 2017}
                % \item {\small J. Ha, A. Kim, \textbf{D. Kim}, C. Kim, J. Park, ``Music Highlight Extraction via Convolutional Recurrent Attention Networks,'' {\em Machine Learning for Music Discovery Workshop in International Conference on Machine Learning} (ICML 2017), Aug, 2017.}
            \end{cvitems}
        }

    \cvacademicentry
        {System Engineering (12 papers)} % Header
        {} % Proceeding
        { % Description(s) bullet points
            \begin{cvitems}
                \item {\small C. Yoon, \textbf{D. Kim}, W. Jung, C. Kang, H. Cha, ``AppScope: Application Energy Metering Framework for Android Smartphone using Kernel Activity Monitoring,'' {\em Proceeding of 2012 USENIX Annual Technical Conference} (USENIX ATC), Jun., 2012.}
                \item {\small \textbf{D. Kim}, N. Jung, Y. Chon, H. Cha, ``Content-Centric Energy Management of Mobile Displays,'' {\em IEEE Transactions on Mobile Computing} (TMC), DOI: 10.1109/TMC.2015.2467393}
            \end{cvitems}
        }
\end{cventries}


% %-------------------------------------------------------------------------------
%	SECTION TITLE
%-------------------------------------------------------------------------------
\cvsection{Summary}


%-------------------------------------------------------------------------------
%	CONTENT
%-------------------------------------------------------------------------------
\begin{cventries}
    \cvsimpleentry
        { % Description(s) bullet points
            \begin{cvitems}
                \item{
                    {\bf Career} \\
                    - Data engineer for a music streaming service \\
                    \techstyle{Understanding the whole processes of music recommendation} \\
                    - System engineer for infrastructures \\
                    \techstyle{Understanding the internal operations of computer systems} \nn
                }
                \item{
                    {\bf Fields}: Cloud computing, data engineering, recommender system, system engineering \nn
                }
                \item{
                    {\bf Programming Languages}: Python, C, Bash, Scala, Java \nn
                }
                \item{
                    {\bf Skills}: Spark, MapReduce, Hive, Presto, HBase, Consul, Nomad, Docker, RabbitMQ, FastText, Annoy, Faiss\nn
                }
                % \item{
                %     {\bf Hobby}: League of Legends, playing acoustic guitar, listening to music
                % }
            \end{cvitems}
        }
\end{cventries}
% %-------------------------------------------------------------------------------
%	SECTION TITLE
%-------------------------------------------------------------------------------
\cvsection{Work Experience}


%-------------------------------------------------------------------------------
%	CONTENT
%-------------------------------------------------------------------------------
\begin{cventries}
    \cventry
        {Music Recommender System} % Title
        {Oct. 2016 - Present} % Date(s)
        {NAVER Corp.,} % Organization
        {Sungnam, Korea}% Location
        { % Corp desc
            Korea's top search portal which of the music service has a membership of over 1.2 million.
        }
        { % Description(s) of tasks/responsibilities
            \begin{cvitems}
                \item {
                    {\bf Data Pipeline} \\
                    - Collecting and cleansing the massive data in Hadoop environments
                        \techstyle{Python, Spark, Hive} \\
                    - Development of a pipeline for audio feature extraction with various trained models
                        \techstyle{Python, Pyarrow, RabbitMQ} \nn
                }
                \item {
                    {\bf Item-to-item Recommendation} \\
                    - Song and artist recommendation which you may also like
                        \techstyle{Gensim, Spark MLLib, Fasttext} \\
                    - Similarity search between millions of items while maintaining up-to-date information
                        \techstyle{Annoy, Faiss, Spark} \\
                    - Modeling musical instrument classifier with audio content data and convolutional neural networks
                        \techstyle{Keras} \\
                    - Song recommendation from natural language based on word embeddings
                        \techstyle{Gensim, Fasttext} \\
                    - Discarding duplicate songs among millions of songs using titles and audio contents
                        \techstyle{Spark, Hive} \nn
                }
                \item {
                    {\bf Playlist Generation \& Continuation} \\
                    - Discover songs not listened yet applying a mixture of collaborative/content-based filtering
                        \techstyle{Spark, Hive} \\
                    - Development of a real-time recommender system applying user's listening feedback
                        \techstyle{Python, Redis} \nn
                }
                \item{
                    {\bf Infrastructure Developments} \\
                    - Design and development of A/B test platform
                        \techstyle{HBase, Spark, Spring Boot} \nn
                }
            \end{cvitems}
        }
\end{cventries}
% % \pagebreak
% %-------------------------------------------------------------------------------
%	SECTION TITLE
%-------------------------------------------------------------------------------
\cvsection{Education}


%-------------------------------------------------------------------------------
%	CONTENT
%-------------------------------------------------------------------------------
\begin{cventries}
    \cveducationentry
        {M.Sc. in Computer Science} % Degree
        {Mar. 2011 - Aug. 2015} % Date(s)
        {Yonsei University, } % Institution
        {Seoul, Korea} % Location
        {one of Korea's top-three universities, considered the most prestigious in the country.} % Institution desc
        {``Content-centric energy management for mobile devices'' {\em (Advisor: Prof. Hojung Cha)}}
        {} %FIXME
        
    \cveducationentry
        {B.S. in Computer Science} % Degree
        {Mar. 2007 - Feb. 2011} % Date(s)
        {Yonsei University} % Institution
        {Seoul, Korea} % Location
        {} % Institution desc
        {} % title paper
        {} %FIXME
\end{cventries}

% %-------------------------------------------------------------------------------
%	SECTION TITLE
%-------------------------------------------------------------------------------
\cvsection{Selected Publications}


%-------------------------------------------------------------------------------
%	CONTENT
%-------------------------------------------------------------------------------
\begin{cventries}
    \cvacademicentry
        {Deep Learning (3 papers)} % Header
        {} % Proceeding
        { % Description(s) bullet points
            \begin{cvitems}
                \item {\small B. Jeon, A. Kim, C. Kim, \textbf{D. Kim}, J. Park, J. Ha, ``Music Emotion Recognition via End-to-End Multimodal Neural Networks,'' {\em Proceeding of the 11th ACM Conference on Recommender Systems} (RecSys 2017), Aug. 2017}
                % \item {\small J. Ha, A. Kim, \textbf{D. Kim}, C. Kim, J. Park, ``Music Highlight Extraction via Convolutional Recurrent Attention Networks,'' {\em Machine Learning for Music Discovery Workshop in International Conference on Machine Learning} (ICML 2017), Aug, 2017.}
            \end{cvitems}
        }

    \cvacademicentry
        {System Engineering (12 papers)} % Header
        {} % Proceeding
        { % Description(s) bullet points
            \begin{cvitems}
                \item {\small C. Yoon, \textbf{D. Kim}, W. Jung, C. Kang, H. Cha, ``AppScope: Application Energy Metering Framework for Android Smartphone using Kernel Activity Monitoring,'' {\em Proceeding of 2012 USENIX Annual Technical Conference} (USENIX ATC), Jun., 2012.}
                \item {\small \textbf{D. Kim}, N. Jung, Y. Chon, H. Cha, ``Content-Centric Energy Management of Mobile Displays,'' {\em IEEE Transactions on Mobile Computing} (TMC), DOI: 10.1109/TMC.2015.2467393}
            \end{cvitems}
        }
\end{cventries}


% %-------------------------------------------------------------------------------
%	SECTION TITLE
%-------------------------------------------------------------------------------
\cvsection{Summary}


%-------------------------------------------------------------------------------
%	CONTENT
%-------------------------------------------------------------------------------
\begin{cventries}
    \cvsimpleentry
        { % Description(s) bullet points
            \begin{cvitems}
                \item{
                    {\bf Career} \\
                    - Data engineer for a music streaming service \\
                    \techstyle{Understanding the whole processes of music recommendation} \\
                    - System engineer for infrastructures \\
                    \techstyle{Understanding the internal operations of computer systems} \nn
                }
                \item{
                    {\bf Fields}: Cloud computing, data engineering, recommender system, system engineering \nn
                }
                \item{
                    {\bf Programming Languages}: Python, C, Bash, Scala, Java \nn
                }
                \item{
                    {\bf Skills}: Spark, MapReduce, Hive, Presto, HBase, Consul, Nomad, Docker, RabbitMQ, FastText, Annoy, Faiss\nn
                }
                % \item{
                %     {\bf Hobby}: League of Legends, playing acoustic guitar, listening to music
                % }
            \end{cvitems}
        }
\end{cventries}
% %-------------------------------------------------------------------------------
%	SECTION TITLE
%-------------------------------------------------------------------------------
\cvsection{Work Experience}


%-------------------------------------------------------------------------------
%	CONTENT
%-------------------------------------------------------------------------------
\begin{cventries}
    \cventry
        {Music Recommender System} % Title
        {Oct. 2016 - Present} % Date(s)
        {NAVER Corp.,} % Organization
        {Sungnam, Korea}% Location
        { % Corp desc
            Korea's top search portal which of the music service has a membership of over 1.2 million.
        }
        { % Description(s) of tasks/responsibilities
            \begin{cvitems}
                \item {
                    {\bf Data Pipeline} \\
                    - Collecting and cleansing the massive data in Hadoop environments
                        \techstyle{Python, Spark, Hive} \\
                    - Development of a pipeline for audio feature extraction with various trained models
                        \techstyle{Python, Pyarrow, RabbitMQ} \nn
                }
                \item {
                    {\bf Item-to-item Recommendation} \\
                    - Song and artist recommendation which you may also like
                        \techstyle{Gensim, Spark MLLib, Fasttext} \\
                    - Similarity search between millions of items while maintaining up-to-date information
                        \techstyle{Annoy, Faiss, Spark} \\
                    - Modeling musical instrument classifier with audio content data and convolutional neural networks
                        \techstyle{Keras} \\
                    - Song recommendation from natural language based on word embeddings
                        \techstyle{Gensim, Fasttext} \\
                    - Discarding duplicate songs among millions of songs using titles and audio contents
                        \techstyle{Spark, Hive} \nn
                }
                \item {
                    {\bf Playlist Generation \& Continuation} \\
                    - Discover songs not listened yet applying a mixture of collaborative/content-based filtering
                        \techstyle{Spark, Hive} \\
                    - Development of a real-time recommender system applying user's listening feedback
                        \techstyle{Python, Redis} \nn
                }
                \item{
                    {\bf Infrastructure Developments} \\
                    - Design and development of A/B test platform
                        \techstyle{HBase, Spark, Spring Boot} \nn
                }
            \end{cvitems}
        }
\end{cventries}


%-------------------------------------------------------------------------------
%	SECTION TITLE
%-------------------------------------------------------------------------------
\cvsection{Summary}


%-------------------------------------------------------------------------------
%	CONTENT
%-------------------------------------------------------------------------------
\begin{cventries}
    \cvsimpleentry
        { % Description(s) bullet points
            \begin{cvitems}
                \item{
                    {\bf Career} \\
                    - Data engineer for a music streaming service \\
                    \techstyle{Understanding the whole processes of music recommendation} \\
                    - System engineer for infrastructures \\
                    \techstyle{Understanding the internal operations of computer systems} \nn
                }
                \item{
                    {\bf Fields}: Cloud computing, data engineering, recommender system, system engineering \nn
                }
                \item{
                    {\bf Programming Languages}: Python, C, Bash, Scala, Java \nn
                }
                \item{
                    {\bf Skills}: Spark, MapReduce, Hive, Presto, HBase, Consul, Nomad, Docker, RabbitMQ, FastText, Annoy, Faiss\nn
                }
                % \item{
                %     {\bf Hobby}: League of Legends, playing acoustic guitar, listening to music
                % }
            \end{cvitems}
        }
\end{cventries}
%-------------------------------------------------------------------------------
%	SECTION TITLE
%-------------------------------------------------------------------------------
\cvsection{Work Experience}


%-------------------------------------------------------------------------------
%	CONTENT
%-------------------------------------------------------------------------------
\begin{cventries}
    \cventry
        {Music Recommender System} % Title
        {Oct. 2016 - Present} % Date(s)
        {NAVER Corp.,} % Organization
        {Sungnam, Korea}% Location
        { % Corp desc
            Korea's top search portal which of the music service has a membership of over 1.2 million.
        }
        { % Description(s) of tasks/responsibilities
            \begin{cvitems}
                \item {
                    {\bf Data Pipeline} \\
                    - Collecting and cleansing the massive data in Hadoop environments
                        \techstyle{Python, Spark, Hive} \\
                    - Development of a pipeline for audio feature extraction with various trained models
                        \techstyle{Python, Pyarrow, RabbitMQ} \nn
                }
                \item {
                    {\bf Item-to-item Recommendation} \\
                    - Song and artist recommendation which you may also like
                        \techstyle{Gensim, Spark MLLib, Fasttext} \\
                    - Similarity search between millions of items while maintaining up-to-date information
                        \techstyle{Annoy, Faiss, Spark} \\
                    - Modeling musical instrument classifier with audio content data and convolutional neural networks
                        \techstyle{Keras} \\
                    - Song recommendation from natural language based on word embeddings
                        \techstyle{Gensim, Fasttext} \\
                    - Discarding duplicate songs among millions of songs using titles and audio contents
                        \techstyle{Spark, Hive} \nn
                }
                \item {
                    {\bf Playlist Generation \& Continuation} \\
                    - Discover songs not listened yet applying a mixture of collaborative/content-based filtering
                        \techstyle{Spark, Hive} \\
                    - Development of a real-time recommender system applying user's listening feedback
                        \techstyle{Python, Redis} \nn
                }
                \item{
                    {\bf Infrastructure Developments} \\
                    - Design and development of A/B test platform
                        \techstyle{HBase, Spark, Spring Boot} \nn
                }
            \end{cvitems}
        }
\end{cventries}


% \pagebreak
%-------------------------------------------------------------------------------
%	SECTION TITLE
%-------------------------------------------------------------------------------
\cvsection{Education}


%-------------------------------------------------------------------------------
%	CONTENT
%-------------------------------------------------------------------------------
\begin{cventries}
    \cveducationentry
        {M.Sc. in Computer Science} % Degree
        {Mar. 2011 - Aug. 2015} % Date(s)
        {Yonsei University, } % Institution
        {Seoul, Korea} % Location
        {one of Korea's top-three universities, considered the most prestigious in the country.} % Institution desc
        {``Content-centric energy management for mobile devices'' {\em (Advisor: Prof. Hojung Cha)}}
        {} %FIXME
        
    \cveducationentry
        {B.S. in Computer Science} % Degree
        {Mar. 2007 - Feb. 2011} % Date(s)
        {Yonsei University} % Institution
        {Seoul, Korea} % Location
        {} % Institution desc
        {} % title paper
        {} %FIXME
\end{cventries}

%-------------------------------------------------------------------------------
%	SECTION TITLE
%-------------------------------------------------------------------------------
\cvsection{Selected Publications}


%-------------------------------------------------------------------------------
%	CONTENT
%-------------------------------------------------------------------------------
\begin{cventries}
    \cvacademicentry
        {Deep Learning (3 papers)} % Header
        {} % Proceeding
        { % Description(s) bullet points
            \begin{cvitems}
                \item {\small B. Jeon, A. Kim, C. Kim, \textbf{D. Kim}, J. Park, J. Ha, ``Music Emotion Recognition via End-to-End Multimodal Neural Networks,'' {\em Proceeding of the 11th ACM Conference on Recommender Systems} (RecSys 2017), Aug. 2017}
                % \item {\small J. Ha, A. Kim, \textbf{D. Kim}, C. Kim, J. Park, ``Music Highlight Extraction via Convolutional Recurrent Attention Networks,'' {\em Machine Learning for Music Discovery Workshop in International Conference on Machine Learning} (ICML 2017), Aug, 2017.}
            \end{cvitems}
        }

    \cvacademicentry
        {System Engineering (12 papers)} % Header
        {} % Proceeding
        { % Description(s) bullet points
            \begin{cvitems}
                \item {\small C. Yoon, \textbf{D. Kim}, W. Jung, C. Kang, H. Cha, ``AppScope: Application Energy Metering Framework for Android Smartphone using Kernel Activity Monitoring,'' {\em Proceeding of 2012 USENIX Annual Technical Conference} (USENIX ATC), Jun., 2012.}
                \item {\small \textbf{D. Kim}, N. Jung, Y. Chon, H. Cha, ``Content-Centric Energy Management of Mobile Displays,'' {\em IEEE Transactions on Mobile Computing} (TMC), DOI: 10.1109/TMC.2015.2467393}
            \end{cvitems}
        }
\end{cventries}



%-------------------------------------------------------------------------------
\end{document}
